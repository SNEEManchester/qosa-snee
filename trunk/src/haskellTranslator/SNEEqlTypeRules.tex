\section{Typing a \SNEEql Query}
\subsection{Typing Rules}
\label{sec_TypingRules}
\texttt{Formally defines the different types of data that queries and operators can produce such as relations, streams of tagged tuples and streams of windows.
Provides formal rules which specify for every possible query what the output type will be.
These will be presented as formal equations here with the corresponding Haskell code left for the Appendix.}

This sections introduces the typing rules which given an AST or DDL element return the type of data represented by this element.

%\input{SNEEqlTypes.lhs}

\subsection{Typing Rules Format}
\label{sec_TypingRulesFormat}

A rule is typeset (and numbered) with the premiss(es) above a horizontal
line and the conclusions below. For Example:

\Rule{
$\epsilon$ = \syn{(Extent ($\epsilon_1$ $\epsilon_2$))},
\\ \environment $\ni$ \typeof{$\epsilon_1$}{$\tau$}
}{
\environment $\ni$ \typeof{$\epsilon$}{$\tau$},
\\ \environment $\ni$ \boundto{$\epsilon_2$}{$\epsilon_1$}
}{rule_Ex1}

In the rules epsilon ($\epsilon$) is used to represent the AST or DDL element being typed while numbered ($\epsilon_1$) or primed ($\epsilon'$) epsilons represent sub elements.
A sub element that plays no active role in the rule is represented as just an underscore.

The font \syn{Element} is used for DDL and AST element constructors,
while the font \typ{Type} is used of type declarations. 

Tau $\tau$ (including its numbered and primed versions) is used to represent a type variable.

\syscat denote the system catalog, and \environment denote an associative environment.
Declarations and type assertions made for a query or sub query only hold for that specific query or sub query.
Environment is therefore scoped with \environment representing the current scope and \environment' a local scope accessible only within this query or subquery.

The statement \typeof{A}{B} denotes the expression the type of A is B.

The $\ni$ symbol represent the notion of supports.

In the above example the line \environment $\ni$ \typeof{$\epsilon$}{$\tau$} says that the environment supports that element $\epsilon$ types to the type represented by $\tau$.

The last line in the example includes the statement \boundto{$\epsilon_2$}{$\epsilon_1$},
which says that the element $\epsilon_2$ is bound to the element $\epsilon_1$.
Bound to is explained in Section \ref{sec_extentListType}

\input{sneeqlEnvironment.lhs}

\setcounter{equation}{0}

% conclusion without premiss => axiom
% premiss without conclusion => integrity constraint

\subsubsection{Bootstrapping the Environment with DDL Statements} 

Rule \ref{rule_syscat} confirms that all declaration made in the DDL are available to any environment.
\Rule{
\syscat $\ni \epsilon$ 
}{
\environment $\ni \epsilon$ 
}{rule_syscat}

\subsubsection{Deriving Types of DDL-Asserted Attributes} 
Rules \ref{rule_BoolToken} to \ref{rule_StringToken} assign a primitive type to attribute names according to their DDL declaration.

\Rule{
\environment $\ni$ \syn{(DeclAtt ($\epsilon$ BoolToken))}
}{
\environment $\ni$ \typeof{$\epsilon$}{\typ{Boolean}}
}{rule_BoolToken}

\Rule{
\environment $\ni$ \syn{(DeclAtt ($\epsilon$ IntToken))}
}{
\environment $\ni$ \typeof{$\epsilon$}{\typ{int}}
}{rule_IntToken}

\Rule{
\environment $\ni$ \syn{(DeclAtt ($\epsilon$ FloatToken))}
}{
\environment $\ni$ \typeof{$\epsilon$}{\typ{float}}
}{rule_FloatToken}

\Rule{
\environment $\ni$ \syn{(DeclAtt ($\epsilon$ StringToken))}
}{
\environment $\ni$ \typeof{$\epsilon$}{\typ{string}}
}{rule_StringToken}

\subsubsection{Typing Literals of Primitive Types}
Rules \ref{rule_BooleanLitType} assign the primitive type \typ{Boolean} to the AST tokens \syn{(True)} and \syn{(False)}.

\Rule{
($\epsilon$ = \syn{(True)} $\vee$ $\epsilon$ = \syn{(False)}
}{
\environment $\ni$ \typeof{$\epsilon$}{\typ{Boolean}}
}{rule_BooleanLitType}

Rules \ref{rule_IntType} to \ref{rule_StringType} assign primitive types to the AST tokens \syn{IntLit}, \syn{FloatLit} and \syn{StringLit} which are wrappers around literals.

\Rule{
$\epsilon$ = \syn{(IntLit($\epsilon_1$))}
}{
\environment $\ni$ \typeof{$\epsilon$}{\typ{int}},
}{rule_IntType}

\Rule{
$\epsilon$ = \syn{(FloatLit($\epsilon_1$))}
}{
\environment $\ni$ \typeof{$\epsilon$}{\typ{float}}
}{rule_FloatType}

\Rule{
$\epsilon$ = \syn{(StringLit($\epsilon_1$))}
}{
\environment $\ni$ \typeof{$\epsilon$}{\typ{string}}
}{rule_StringType}

\subsubsection{Deriving Types of DDL-Asserted and Derived Extents}
Rules \ref{rule_RelationExt} to \ref{rule_WindowExt} type DeclExt declaration according to the type token included.
In all cases the inner tuple type is determined by getting the types of the attribute based on the conclusions of applying rules \ref{rule_BoolToken} to \ref{rule_StringToken} on the attribute names [$\epsilon_1$,
  \mdots, $\epsilon_n$].

Rule \ref{rule_RelationExt} is for Relations such as stored tables, relational sub queries and views resulting in relations.
\Rule{
\environment $\ni$ \syn{(DeclExt ($\epsilon$ $\epsilon'$ [$\epsilon_1$,
  \mdots, $\epsilon_n$]))}, 
\\ $\epsilon' \in$ \{\syn{StoredToken, RelationView}\}, 
\\ \environment $\ni$ \typeof{$\epsilon_1$}{$\tau_1$}, \mdots,
  \environment $\ni$ \typeof{$\epsilon_n$}{$\tau_n$} 
}{
\environment $\ni$ \typeof{$\epsilon$}{\bagof{\tupleof{$\tau_1$, \mdots, $\tau_n$}}}
}{rule_RelationExt}

Rule \ref{rule_StreamExt} is for streams of tagged tuples which could come from sensed/ acquistional sources, push streams, as well as sub queries and views resulting in a stream of tagged tuples.
\Rule{
\environment $\ni$ \syn{(DeclExt ($\epsilon$ $\epsilon'$ [$\epsilon_1$,
  \mdots, $\epsilon_n$]))}, 
\\ $\epsilon' \in$ \{\syn{SensedToken, PushedToken, StreamView}\}, 
\\ \environment $\ni$ \typeof{$\epsilon_1$}{$\tau_1$}, \mdots,
  \environment $\ni$ \typeof{$\epsilon_n$}{$\tau_n$} 
}{
\environment $\ni$ \typeof{$\epsilon$}{\streamof{\tupleof{\typ{int}, \typ{int}, \tupleof{$\tau_1$, \mdots, $\tau_n$}}}}
}{rule_StreamExt}

Rule \ref{rule_WindowExt} is for streams of windows which could only come from sub queries and views.

\Rule{
\environment $\ni$ \syn{(DeclExt ($\epsilon$ $\epsilon'$ [$\epsilon_1$,
  \mdots, $\epsilon_n$]))}, 
\\ $\epsilon' \in$ \{\syn{WindowView}\}, 
\\ \environment $\ni$ \typeof{$\epsilon_1$}{$\tau_1$}, \mdots,
  \environment $\ni$ \typeof{$\epsilon_n$}{$\tau_n$}
}{
\environment $\ni$ \typeof{$\epsilon$}{\streamof{\tupleof{\typ{int}, \bagof{\tupleof{$\tau_1$, \mdots, $\tau_n$}}}}}
}{rule_WindowExt}

\subsection{Typing a Query}
\label{sec_wholeQueryType}

\subsubsection{Typing a FullQuery}

To distinguish a full query from one that serves as a sub element in DStream, IStream, RStream, Sub Queries and later views, a \syn{FullQuery} wrapper is placed around the outermost \syn{Query}. 
This allows the translation rule \ref{rule_TranslateFullQuery} to apply the final \sem{Deliver} operator.
As \sem{Deliver} has no affect on the type, the typing rule for \syn{FullQuery} is just a pass through where the type of the inner query is the type of the \syn{FullQuery}.
\Rule{
$\epsilon$ = \syn{(FullQuery ($\epsilon_1$)},
\\ \environment $\ni$ \typeof{$\epsilon_1$}{$\tau$}
}{
\environment $\ni$ \typeof{$\epsilon$}{$\tau$}
}{rule_FullQueryType}

Typing a \syn{Query} introduces a local environment which has access to
all declarations in the system catalog.  By using a local environment the bindings and type assertions 
made on the sub elements of the query are only available locally to this query or subquery.

The process consists of
typing the \syn{ExtentList} to determine the collection type of the
query and typing the \syn{ProjectList} or \syn{AggregationList} to
determine the type of every tuple in the collection.  The \syn{Predicate}
is typed to \typ{Boolean} to verify it is correct.  

Because the query could be a subquery or view, the rules declare an extent with the query name, the appropriate \syn{TypeToken} and the \syn{AttributeName}s.
The \syn{Query} rules also assert the types associated with each of the attribute names in the outer environment.

\subsubsection{Typing a Relational Query}
Rule \ref{rule_RelationType} says that both \syn{query} and \syn{AggregationQuery} type to a bag of tuples as long as their \syn{Extent}s $\epsilon_2$ type to a bag of tuples.

\Rule{
( $\epsilon$ = \syn{(Query ($\epsilon_1$ $\epsilon_2$ $\epsilon_3$  $\epsilon_4$))},
\\ $\vee$ $\epsilon$ = \syn{(AggregationQuery ($\epsilon_1$ $\epsilon_2$ $\epsilon_3$  $\epsilon_4$))}),
\\  \environment' $\ni$ \typeof{$\epsilon_2$}{\bagof{\tupleof{\_}}},
\\  $\epsilon_1$ = \syn{[(Projection ($\epsilon'_1$ $\epsilon''_1$)), \mdots, (Projection ($\epsilon'_n$ $\epsilon''_n$))]},
\\  \environment' $\ni$ \typeof{$\epsilon'_1$}{$\tau_1$}, \mdots, \environment $\ni$ \typeof{$\epsilon'_n$}{$\tau_n$},
\\  \environment' $\ni$ \typeof{$\epsilon_3$}{\typ{Boolean}}
}{
\environment $\ni$ \typeof{$\epsilon$}{\bagof{\tupleof{$\tau_1$, \mdots, $\tau_n$}}},
\\ \environment $\ni$ \typeof{$\epsilon''_1$}{$\tau_1$},  \mdots,
    \environment $\ni$ \typeof{$\epsilon''_n$}{$\tau_n$},
\\ \environment $\ni$ \syn{(DeclExt ($\epsilon_4$ RelationView [$\epsilon''_1$, \mdots, $\epsilon''_n$]))}
\\ 
}{rule_RelationType}

%The rule for relational queries with aggregation (rule \ref{rule_RelationAggregationType}) is similar to the rule for no aggregate relational queries except that the tuples type comes from the list of \syn{Aggregation}s rather than the \syn{Projection}s.
%\Rule{
%$\epsilon$ = \syn{(AggregationQuery ($\epsilon_1$ $\epsilon_2$ $\epsilon_3$  $\epsilon_4$))},
%\\  \environment' $\ni$ \typeof{$\epsilon_2$}{\bagof{\tupleof{\_}}},
%\\  $\epsilon_1$ = \syn{[(Aggregation ($\epsilon'_1$ $\epsilon''_1$)), \mdots, (Aggregation ($\epsilon'_n$ $\epsilon''_n$))]},
%\\  \environment' $\ni$ \typeof{$\epsilon_1$}{$\tau_1$}, \mdots, \environment $\ni$ \typeof{$\epsilon_n$}{$\tau_n$},
%\\  \environment' $\ni$ \typeof{$\epsilon_3$}{\typ{Boolean}}
%}{
%\environment $\ni$ \typeof{$\epsilon$}{\bagof{\tupleof{$\tau_1$, \mdots, $\tau_n$}}},
%\\ \environment $\ni$ \typeof{$\epsilon''_1$}{$\tau_1$},  \mdots,
%    \environment $\ni$ \typeof{$\epsilon''_n$}{$\tau_n$},
%\\ \environment $\ni$ \syn{(DeclExt ($\epsilon_4$ RelationView [$\epsilon''_1$, \mdots, $\epsilon''_n$]))}
%}{rule_RelationAggregationType}

\subsubsection{Typing a Window-Stream Query}
\SNEEql allows queries that return Streams of Windows.
The most common use will be as inner queries for \syn{IStream}, \syn{Dstream} and \syn{Rstream} see rule \ref{rule_DStreamType}, but they could also be subqueries and views.
There is no semantic reason why full queries could not return Window-Streams as long as the recipient can handle data which is not in the first normal form.

A \syn{Query} or \syn{AggregationQuery} returns a streams of windows when the \syn{ExtentList} $\epsilon_2$ returns a Stream of windows, for example because it contains window definitions (see rule \ref{rule_WindowedExtentType}) or a join over windowed extents (see rules \ref{rule_ProdWinType} and \ref{rule_ProdRelWin}).

The type of the inner tuple is determined by typing the \syn{Projection} list just like rule \ref{rule_RelationType}.

\Rule{
($\epsilon$ = \syn{(Query ($\epsilon_1$ $\epsilon_2$ $\epsilon_3$  $\epsilon_4$))},
\\ $\vee$ $\epsilon$ = \syn{(AggregationQuery ($\epsilon_1$ $\epsilon_2$ $\epsilon_3$  $\epsilon_4$))}),
\\  \environment' $\ni$ \typeof{$\epsilon_2$}{\streamof{\tupleof{$\tau_0$, \bagof{\tupleof{\_}}}}},
\\  $\epsilon_1$ = \syn{[(Projection ($\epsilon'_1$ $\epsilon''_1$)), \mdots, (Projection ($\epsilon'_n$ $\epsilon''_n$))]},
\\  \environment' $\ni$ \typeof{$\epsilon_1$}{$\tau_1$}, \mdots, \environment $\ni$ \typeof{$\epsilon_n$}{$\tau_n$},
\\  \environment' $\ni$ \typeof{$\epsilon_3$}{\typ{Boolean}}
}{
\environment $\ni$ \typeof{$\epsilon$}{\streamof{\tupleof{$\tau_0$, \bagof{\tupleof{$\tau_1$, \mdots, $\tau_n$}}}}},
\\ \environment $\ni$ \typeof{$\epsilon''_1$}{$\tau_1$},  \mdots,
    \environment $\ni$ \typeof{$\epsilon''_n$}{$\tau_n$},
\\ \environment $\ni$ \syn{(DeclExt ($\epsilon_4$ WindowView [$\epsilon''_1$, \mdots, $\epsilon''_n$]))}
}{rule_WindowType}

%Typing windowed queries with aggregation is similar to typing window queries without except that the type of the inner tuple comes from the \syn{Aggreagtion} list.
%\Rule{
%$\epsilon$ = \syn{(AggregationQuery ($\epsilon_1$ $\epsilon_2$ $\epsilon_3$  $\epsilon_4$))},
%\\  \environment' $\ni$ \typeof{$\epsilon_2$}{\streamof{\tupleof{$\tau_0$, \bagof{\tupleof{\_}}}}},
%\\  $\epsilon_1$ = \syn{[(Aggregation ($\epsilon'_1$ $\epsilon''_1$)), \mdots, (Aggregation ($\epsilon'_n$ $\epsilon''_n$))]},
%\\  \environment' $\ni$ \typeof{$\epsilon_1$}{$\tau_1$}, \mdots, \environment $\ni$ \typeof{$\epsilon_n$}{$\tau_n$},
%\\  \environment' $\ni$ \typeof{$\epsilon_3$}{\typ{Boolean}}
%}{
%\environment $\ni$ \typeof{$\epsilon$}{\streamof{\tupleof{$\tau_0$, \bagof{\tupleof{$\tau_1$, \mdots, $\tau_n$}}}}},
%\\ \environment $\ni$ \typeof{$\epsilon''_1$}{$\tau_1$},  \mdots,
%    \environment $\ni$ \typeof{$\epsilon''_n$}{$\tau_n$},
%\\ \environment $\ni$ \syn{(DeclExt ($\epsilon_4$ WindowView [$\epsilon''_1$, \mdots, $\epsilon''_n$]))}
%}{rule_WindowAggregationType}

\subsubsection{Typing a Group By Query}
A \syn{GroupByQuery} has two extra parameters compared to normal queries.
The \syn{GroupByLIst} $\epsilon_4$ which represents the list of \syn{Attribute}s declared in the Group By clause.
This must be validated by Rule \ref{rule_RelationGroupByType} which also declares the group by attributes in the local environment used for typing the other tokens.
A second \syn{BooleanExpression} $\epsilon_5$ represents the HAVING clause. 
This too has to type to \typ{Boolean} for the query to be valid.

Rule \ref{rule_RelationGroupByType} for a relational \syn{GroupByQuery} includes all the premisses and conclusions from Rule \ref{rule_RelationType} with the \syn{ExtentName} now being $\epsilon_6$.
\Rule{
$\epsilon$ = \syn{(GroupByQuery ($\epsilon_1$ $\epsilon_2$ $\epsilon_3$ $\epsilon_4$ $\epsilon_6$ $\epsilon_5$))},
\\  \environment' $\ni$ \typeof{$\epsilon_2$}{\bagof{\tupleof{\_}}},
\\  \environment' $\ni$ \typeof{$\epsilon_4$}{\_ },
\\  $\epsilon_1$ = \syn{[(Projection ($\epsilon'_1$ $\epsilon''_1$)), \mdots, (Projection ($\epsilon'_n$ $\epsilon''_n$))]},
\\  \environment' $\ni$ \typeof{$\epsilon'_1$}{$\tau_1$}, \mdots, \environment $\ni$ \typeof{$\epsilon'_n$}{$\tau_n$},
\\  \environment' $\ni$ \typeof{$\epsilon_3$}{\typ{Boolean}},
\\  \environment' $\ni$ \typeof{$\epsilon_5$}{\typ{Boolean}}
}{
\environment $\ni$ \typeof{$\epsilon$}{\bagof{\tupleof{$\tau_1$, \mdots, $\tau_n$}}},
\\ \environment $\ni$ \typeof{$\epsilon''_1$}{$\tau_1$},  \mdots,
    \environment $\ni$ \typeof{$\epsilon''_n$}{$\tau_n$},
\\ \environment $\ni$ \syn{(DeclExt ($\epsilon_6$ RelationView [$\epsilon''_1$, \mdots, $\epsilon''_n$]))}
}{rule_RelationGroupByType}

Rule \ref{rule_WindowGroupByType} for a window \syn{GroupByQuery} includes all the premisses and conclusions from Rule \ref{rule_WindowType} with the \syn{ExtentName} now being $\epsilon_6$.
\Rule{
$\epsilon$ = \syn{(GroupByQuery ($\epsilon_1$ $\epsilon_2$ $\epsilon_3$ $\epsilon_4$ $\epsilon_5$))},
\\  \environment' $\ni$ \typeof{$\epsilon_2$}{\streamof{\tupleof{$\tau_0$, \bagof{\tupleof{\_}}}}},
\\  \environment' $\ni$ \typeof{$\epsilon_4$}{\_ },
\\  $\epsilon_1$ = \syn{[(Projection ($\epsilon'_1$ $\epsilon''_1$)), \mdots, (Projection ($\epsilon'_n$ $\epsilon''_n$))]},
\\  \environment' $\ni$ \typeof{$\epsilon'_1$}{$\tau_1$}, \mdots, \environment $\ni$ \typeof{$\epsilon'_n$}{$\tau_n$},
\\  \environment' $\ni$ \typeof{$\epsilon_3$}{\typ{Boolean}},
\\  \environment' $\ni$ \typeof{$\epsilon_5$}{\typ{Boolean}}
}{
\environment $\ni$ \typeof{$\epsilon$}{\streamof{\tupleof{$\tau_0$, \bagof{\tupleof{$\tau_1$, \mdots, $\tau_n$}}}}},
\\ \environment $\ni$ \typeof{$\epsilon''_1$}{$\tau_1$},  \mdots,
    \environment $\ni$ \typeof{$\epsilon''_n$}{$\tau_n$},
\\ \environment $\ni$ \syn{(DeclExt ($\epsilon_6$ WindowView [$\epsilon''_1$, \mdots, $\epsilon''_n$]))}
}{rule_WindowGroupByType}

\subsubsection{Typing Tuple-Stream Queries}

Most queries in \SNEEql will return stream of tuples. 
For simple queries (those without aggregation, group by or joins) a simple tuple stream query will do.
Notice how there is no rule for \syn{AggregationQuery} or \syn{GroupByQuery} that allow the extents and therefor the whole query to be of type stream of tuples.

\Rule{
$\epsilon$ = \syn{(Query ($\epsilon_1$ $\epsilon_2$ $\epsilon_3$  $\epsilon_4$))},
\\  \environment' $\ni$ \typeof{$\epsilon_2$}{\streamof{\tupleof{$\tau_0$, $\tau_0'$, \tupleof{\_}}}},
\\  $\epsilon_1$ = \syn{[(Projection ($\epsilon'_1$ $\epsilon''_1$)), \mdots, (Projection ($\epsilon'_n$ $\epsilon''_n$))]},
\\  \environment' $\ni$ \typeof{$\epsilon_1$}{$\tau_1$}, \mdots, \environment $\ni$ \typeof{$\epsilon_n$}{$\tau_n$},
\\  \environment' $\ni$ \typeof{$\epsilon_3$}{\typ{Boolean}}
}{
\environment $\ni$ \typeof{$\epsilon$}{\streamof{\tupleof{$\tau_0$, $\tau_0'$, \tupleof{$\tau_1$, \mdots, $\tau_n$}}}},
\\ \environment $\ni$ \typeof{$\epsilon''_1$}{$\tau_1$},  \mdots,
    \environment $\ni$ \typeof{$\epsilon''_n$}{$\tau_n$},
\\ \environment $\ni$ \syn{(DeclExt ($\epsilon_4$ StreamView [$\epsilon''_1$, \mdots, $\epsilon''_n$]))}
}{rule_StreamType}

\subsubsection{Typing IStream, DStream, RStream Queries}
Another way to get a tuple stream result is to wrap a window stream query (See rule \ref{rule_WindowType}
% and \ref{rule_WindowAggregationType}
) with an IStream, DStream or RStream.
The rule \ref{rule_DStreamType} states that the window stream output of the inner query will be converted to a stream of tagged tuples.

\Rule{
(($\epsilon$ = \syn{(DStream ($\epsilon_1$))}) $\vee$ ($\epsilon$ = \syn{(IStream ($\epsilon_1$ ))}) 
\\  $\vee$ ($\epsilon$ = \syn{(RStream ($\epsilon_1$))})),
\\  \environment $\ni$ \typeof{$\epsilon_1$}{\streamof{\tupleof{$\tau_0$, \bagof{\tupleof{$\tau_1$, \mdots, $\tau_n$}}}}},
}{
\environment $\ni$ \typeof{$\epsilon$}{\streamof{\tupleof{$\tau_0$, \typ{int}, \tupleof{$\tau_1$, \mdots, $\tau_n$}}}}
}{rule_DStreamType}

\subsection{Typing the Extents in a Query}
\label{sec_extentListType}
The collection type returned by the query is determined by the collection type returned by the extent or the list of extents.
The typing rules for extents also return the attributes within the collection type both for completion and to verify the correctness of the extent declarations.

\SNEEql uses \syn{LocalName}s for all extents used in the query including sub queries. 
This allows self joins and disambiguates \syn{AttributeName} clashes in a consistent way.

The rules for typing extents binds each \syn{LocalName} to an \syn{ExtentName}.

Rule \ref{rule_AttributeType} which types the \syn{Attribute} expression verifies that the \syn{LocalName} used there is bound to an \syn{ExtentName}. 
This verifies that the SELECT and PROJECT clauses to only use extents declared in the FROM clause.

For example given the query:

\dsyn{
      select e.name as n 
  \lf from   \> emp as e 
  \lf where  \> e.name = 'john'
}

The conclusion would then be \boundto{(\syn{LocalName} e)}{(\syn{ExtentName} emp)} which allow the typing rule for attributes (Rule \ref{rule_AttributeType}) to use e to access the extent declaration made for emp and also to confirm that extent emp is used in the from statement.

Formally the declaration \boundto{A}{B} is only valid is A is of type AST element \syn{LocalName} and B is of type AST element \syn{ExtentName}. Because A and B are different AST elements bound to is not reflexive, transitive, associative or commutative. 
The bound to declaration is a many to one declaration where more than one \syn{LocalName} can be bound to a single \syn{ExtentName} as would be the case in a self join query.
Each \syn{LocalName} must be unique with a single query or sub query.
However there is restriction on using a string within a \syn{LocalName} that has already been used as an \syn{ExtentName}.
In fact if no local name is provided the grammar will reuse the inner string of the \syn{ExtentName} as the inner string of the \syn{LocalName}.
\subsubsection{Typing an Extent}
 
Rule \ref{rule_ExtentType} types an extent declaration based on the \syn{ExtentName} which in turn will have been typed based on the DDL \syn{DeclExt} declaration or during the typing of the sub \syn{Query}.

\Rule{
$\epsilon$ = \syn{(Extent $\epsilon_1$ $\epsilon_2$)},
\\ \environment $\ni$ \typeof{$\epsilon_1$}{$\tau$}
}{
\environment $\ni$ \typeof{$\epsilon$}{$\tau$},
\\ \environment $\ni$ \boundto{$\epsilon_2$}{$\epsilon_1$}
}{rule_ExtentType}

Rule \ref{rule_SubqueryType} shows that when the query includes a \syn{SubQuery} the extent type is determined by the type of the sub query. To allow rule \ref{rule_AttributeType} to access the \syn{DeclExt} declaration made as a conclusion of typing the inner query, rule \ref{rule_SubqueryType} binds the \syn{Localname} $\epsilon_3$ to the \syn{ExtentName} $\epsilon_2$. The parser ensures that the \syn{ExtentName} $\epsilon_2$ is the same as that used in the inner \syn{Query} $\epsilon_1$.
\Rule{
$\epsilon$ = \syn{(Subquery ($\epsilon_1$ $\epsilon_2$ $\epsilon_3$))},
\\ \environment $\ni$ \typeof{$\epsilon_1$}{$\tau$}
}{
\environment $\ni$ \typeof{$\epsilon$}{$\tau$}
\\ \environment $\ni$ \boundto{$\epsilon_3$}{$\epsilon_2$}
}{rule_SubqueryType}

Rule \ref{rule_WindowedExtentType} shows that the type of a windowed extent is the type of the inner extent $\epsilon_1$, which should be a stream of tagged tuples, converted to a stream of windows. 
The actual window scope rules do not affect the type.
\Rule{
$\epsilon$ = \syn{(WindowedExtent ($\epsilon_1$ \_ ))},
\\ \environment $\ni$ \typeof{$\epsilon_1$}{$\tau$},
\\ $\tau$ = \typ{\streamof{\tupleof{\typ{int}, \typ{int}, \tupleof{$\tau_1$, \mdots, $\tau_n$}}}}
}{
\environment $\ni$ \typeof{$\epsilon$}{\streamof{\tupleof{\typ{int}, \bagof{\tupleof{$\tau_1$, \mdots, $\tau_n$}}}}},
}{rule_WindowedExtentType}

Rule \ref{rule_SingleExtentType} confirms that if there is only one extent in the extent list then the type of that extent is the type of the whole list.
\Rule{
$\epsilon$ = \syn{(ExtentList [$\epsilon_1$])},
\\  \environment $\ni$ \typeof{$\epsilon_1$}{$\tau$}
}{
\environment $\ni$ \typeof{$\epsilon$}{\bagof{$\tau$}}
}{rule_SingleExtentType}

\subsubsection{Product of Relations}

Rule \ref{rule_ProdRelType} states that an \syn{ExtentList}, where all
extents are of collection type relation, has collection type relation
over the concatenation of all the attributes in the relations.

\Rule{
$\epsilon$ = \syn{(ExtentList [$\epsilon_1$, \mdots, $\epsilon_n$])},
\\  \environment $\ni$ \typeof{$\epsilon_1$}{$\tau_1$}, \mdots, \environment $\ni$ \typeof{$\epsilon_n$}{$\tau_n$},
\\  $\tau_1$ = \typ{\bagof{\tupleof{$\tau_{11}$, \mdots, $\tau_{1m}$}}},
\\  $\vdots$
\\  $\tau_n$ = \typ{\bagof{\tupleof{$\tau_{n1}$, \mdots, $\tau_{np}$}}},
\\  $\tau$ = \typ{\tupleof{$\tau_{11}$, \mdots, $\tau_{1m}$, \mdots, $\tau_{n1}$, \mdots, $\tau_{np}$}},
}{
\environment $\ni$ \typeof{$\epsilon$}{\bagof{$\tau$}}
}{rule_ProdRelType}

\subsubsection{Product of Window Streams}

Rule \ref{rule_ProdWinType} states that an \syn{ExtentList}, where all
extents are of collection type window-stream, has collection type
window-stream over the concatenation of all the attributes in the
windows.

\Rule{
$\epsilon$ = \syn{(ExtentList [$\epsilon_1$, \mdots, $\epsilon_n$])},
\\  \environment $\ni$ \typeof{$\epsilon_1$}{$\tau_1$}, \mdots, \environment $\ni$ \typeof{$\epsilon_n$}{$\tau_n$}
\\  $\tau_1$ = \typ{\streamof{\tupleof{\typ{int}, \bagof{\tupleof{$\tau_{11}$, \mdots, $\tau_{1p}$}}}}},
\\  $\vdots$
\\  $\tau_n$ = \typ{\streamof{\tupleof{\typ{int}, \bagof{\tupleof{$\tau_{n1}$, \mdots, $\tau_{nq}$}}}}},
\\  $\tau$ = \typ{\tupleof{$\tau_{11}$, \mdots, $\tau_{1p}$, \mdots, $\tau_{n1}$, \mdots, $\tau_{nq}$}}
}{
\environment $\ni$ \typeof{$\epsilon$}{\streamof{\tupleof{\typ{int}, \bagof{$\tau$}}}}
}{rule_ProdWinType}

\subsubsection{Product of Multiple Relations and a Stream}

Rule \ref{rule_ProdRelStreamType} states that an \syn{ExtentList},
where all extents but one have collection type relation with the last
being of collection type tuple-stream, has collection type tuple-stream
over the concatenation of all the attributes in the relations and the
stream.

\Rule{
$\epsilon$ = \syn{(ExtentList [$\epsilon_1$, \mdots, $\epsilon_n$])},
\\  \environment $\ni$ \typeof{$\epsilon_1$}{$\tau_1$}, \mdots, \environment $\ni$ \typeof{$\epsilon_n$}{$\tau_n$}
\\  $\tau_1$ = \typ{\streamof{\tupleof{\typ{int}, \typ{int}, \tupleof{$\tau_{11}$, \mdots, $\tau_{1p}$}}}},
\\  $\tau_2$ = \typ{\bagof{\tupleof{$\tau_{21}$, \mdots, $\tau_{2q}$}}},
\\  $\vdots$
\\  $\tau_{n}$ = \typ{\bagof{\tupleof{$\tau_{n1}$, \mdots, $\tau_{nr}$}}},
\\ $\tau$ = \typ{\tupleof{$\tau_{11}$, \mdots, $\tau_{1p}$, $\tau_{21}$, \mdots, $\tau_{2q}$,  \mdots, $\tau_{n1}$, \mdots, $\tau_{nr}$}}
}{
\environment $\ni$ \typeof{$\epsilon$}{\streamof{\tupleof{\typ{int}, \typ{int}, $\tau$}}}
}{rule_ProdRelStreamType}

\subsubsection{Product of Multiple Relations and Window-Streams}

Rule \ref{rule_ProdRelWin} states that an \syn{ExtentList}, where some
extents are of collection type relation and the rest are of collection
type window-stream, has collection type window-stream over the
concatenation of all the attributes in the relations and the windows.

\Rule{
$\epsilon$ = \syn{(ExtentList [$\epsilon_1$, \mdots, $\epsilon_n$])},
\\  \environment $\ni$ \typeof{$\epsilon_1$}{$\tau_1$}, \mdots, \environment $\ni$ \typeof{$\epsilon_n$}{$\tau_n$},
\\  $\tau_1$ = \typ{\streamof{\tupleof{\typ{int}, \bagof{\tupleof{$\tau_{11}$, \mdots, $\tau_{1p}$}}}}},
\\  $\vdots$
\\  $\tau_{n}$ = \typ{\streamof{\tupleof{\typ{int}, \bagof{\tupleof{$\tau_{n1}$, \mdots, $\tau_{nq}$}}}}},
\\  $\tau_{n+1}$ = \typ{\bagof{\tupleof{$\tau_{(n+1)1}$, \mdots, $\tau_{(n+1)r}$}}},
\\  $\vdots$
\\  $\tau_{n+m}$ = \typ{\typ{int}, \bagof{\tupleof{$\tau_{(n+m)1}$, \mdots, $\tau_{(n+m)s}$}}},
\\  \hspace{-0.75cm}$\tau$ = \typ{\tupleof{$\tau_{11}$, \mdots, $\tau_{1p}$, \mdots, $\tau_{n1}$, \mdots, $\tau_{nq}$, $\tau_{(n+1)1}$, \mdots, $\tau_{(n+1)r}$, \mdots, $\tau_{(n+m)1}$, \mdots, $\tau_{(n+m)s}$}}
}{
\environment $\ni$ \typeof{$\epsilon$}{\streamof{\tupleof{\typ{int}, \bagof{$\tau$}}}}
}{rule_ProdRelWin}
%Why not many rels and many windows?

\subsection{Typing the Projected Attributes in a Query}

\subsubsection{Typing Attributes}
Rules \ref{rule_AttributeType} and \ref{rule_TupleAttributeType} determine both the type of an attribute but also that the attribute can be used.

The type associated with the \syn{AttributeName} $\epsilon_2$ will already have been determined by a DDL declaration 
(Rules \ref{rule_BoolToken} to \ref{rule_StringToken}) or by the typing of \syn{Query} used as a sub query (Rules \ref{rule_RelationType} to \ref{rule_StreamType}).

The second role of rule \ref{rule_AttributeType} is to verify the validity.
The first step is to check that the \syn{LocalName} $\epsilon_1$ has been bound to an \syn{ExtentName} $\epsilon_3$ by either the typing of an \syn{Extent} (rule \ref{rule_ExtentType}) or by typing a \syn{SubQuery} (rule \ref{rule_SubqueryType}).
The next step is to obtain the \syn{DeclExt} declaration from the local environment which will have been placed there either because it was in the the system catalogue (Rule \ref{rule_syscat}) or by the typing of a sub \syn{Query} (Rules \ref{rule_RelationType} to \ref{rule_StreamType}).
The final step it to confirm that the \syn{AttributeName} $\epsilon_2$ is actually in the list of \syn{AttributeName}s [$\epsilon'_1$, \mdots, $\epsilon'_n$] found in that \syn{DeclExt}.

Attribute values found within aggregates and therefore within a \syn{Collection} or attributes from the WHERE clause will be be the AST type \syn{TupleAttribute} as they can never refer to the GROUP BY list.
\Rule{
$\epsilon$ = \syn{(TupleAttribute ($\epsilon_1$ $\epsilon_2$))},
\\ \environment $\ni$ \typeof{$\epsilon_2$}{$\tau$},
\\ \environment $\ni$ \boundto{$\epsilon_1$}{$\epsilon_3$},
\\ \environment $\ni$ \syn{(DeclExt ($\epsilon_3$ \_ [$\epsilon'_1$, \mdots, $\epsilon'_n$]))}, 
\\ $\epsilon_2 \in \{\epsilon'_1$, \mdots, $\epsilon'_n$\}
}{
\environment $\ni$ \typeof{$\epsilon$}{$\tau$}
}{rule_TupleAttributeType}

An attribute not in an aggregate will be a plain \syn{Attribute}. 
In addition to check done in rule \ref{rule_TupleAttributeType} the rule must check that either the attribute is in the group by statement or that there is no group by statement.

For \syn{GroupByQuery} the rules(\ref{rule_RelationGroupByType} and \ref{rule_WindowGroupByType}) will have inserted a \syn{DeclGroupBy} declaration into the Environment. 
Only attributes in this declaration are allowed outside of an aggregate.

\Rule{
$\epsilon$ = \syn{(Attribute ($\epsilon_1$ $\epsilon_2$))},
\\ \environment $\ni$ \typeof{$\epsilon_2$}{$\tau$},
\\ \environment $\ni$ \boundto{$\epsilon_1$}{$\epsilon_3$},
\\ \environment $\ni$ \syn{(DeclExt ($\epsilon_3$ \_ [$\epsilon'_1$, \mdots, $\epsilon'_n$]))}, 
\\ $\epsilon_2 \in \{\epsilon'_1$, \mdots, $\epsilon'_n$\},
\\ \environment $\ni$ \syn{(DeclGroupBy ([$\epsilon''_1$, \mdots, $\epsilon''_n$]))}, 
\\  $\epsilon_2 \in \{\epsilon''_1$, \mdots, $\epsilon''_n$\} 
}{
\environment $\ni$ \typeof{$\epsilon$}{$\tau$}
}{rule_AttributeInGroupType}

For a normal \syn{Query} or an \syn{AggregationQuery} where there is no Group By clause the will be no \syn{DeclGroupBy} in the environment. 

\Rule{
$\epsilon$ = \syn{(Attribute ($\epsilon_1$ $\epsilon_2$))},
\\ \environment $\not\ni$ \syn{(DeclGroupBy ([$\epsilon''_1$, \mdots, $\epsilon''_n$]))},
\\ \environment $\ni$ \typeof{$\epsilon_2$}{$\tau$},
\\ \environment $\ni$ \boundto{$\epsilon_1$}{$\epsilon_3$},
\\ \environment $\ni$ \syn{(DeclExt ($\epsilon_3$ \_ [$\epsilon'_1$, \mdots, $\epsilon'_n$]))}, 
\\ $\epsilon_2 \in \{\epsilon'_1$, \mdots, $\epsilon'_n$\},
}{
\environment $\ni$ \typeof{$\epsilon$}{$\tau$}
}{rule_AttributeType}

\subsubsection{Typing Arithmetic Expressions} 
Rule \ref{rule_AddType} states that atomic arithmetic operations
such as \syn{Add}, \syn{Minus}, \syn{Multiply} or \syn{Divide} applied
to arguments of the same numerical type \typ{int} yield a result of that type.
\Rule{
   ($\epsilon$ = \syn{(Add      ($\epsilon_1$ $\epsilon_2$))} $\vee$
    $\epsilon$ = \syn{(Minus    ($\epsilon_1$ $\epsilon_2$))} $\vee$ 
\\  $\epsilon$ = \syn{(Multiply ($\epsilon_1$ $\epsilon_2$))} $\vee$ 
    $\epsilon$ = \syn{(Divide   ($\epsilon_1$ $\epsilon_2$))}),
\\  (\environment $\ni$ \typeof{$\epsilon_1$}{$\tau$} $\wedge$ \environment $\ni$ \typeof{$\epsilon_2$}{$\tau$})
\\ (($\tau$ = \typ{int}) $\vee$ (($\tau$ = \typ{float})
}{
\environment $\ni$ \typeof{$\epsilon$}{{$\tau$}}
}{rule_AddType}

Rule \ref{rule_AddMixType} states that atomic arithmetic operations
such as \syn{Add}, \syn{Minus}, \syn{Multiply} or \syn{Divide} applied
to an \typ{Int} and a \typ{Float} yields a result of type \typ{float}.
This assumes an implied type casting of the \typ{int} value to a \typ{float} value.

\Rule{
   ($\epsilon$ = \syn{(Add      ($\epsilon_1$ $\epsilon_2$))} $\vee$
    $\epsilon$ = \syn{(Minus    ($\epsilon_1$ $\epsilon_2$))} $\vee$ 
\\  $\epsilon$ = \syn{(Multiply ($\epsilon_1$ $\epsilon_2$))} $\vee$ 
    $\epsilon$ = \syn{(Divide   ($\epsilon_1$ $\epsilon_2$))}),
\\  ((\environment $\ni$ \typeof{$\epsilon_1$}{\typ{int}}   $\wedge$ \environment $\ni$ \typeof{$\epsilon_2$}{\typ{float}}) $\vee$
\\   (\environment $\ni$ \typeof{$\epsilon_1$}{\typ{float}} $\wedge$ \environment $\ni$ \typeof{$\epsilon_2$}{\typ{int}})  
}{
\environment $\ni$ \typeof{$\epsilon$}{{\typ{float}}}
}{rule_AddMixType}

Rule \ref{rule_NegateType} states that the arithmetic operation \syn{Negate} applied
to a numerical type is that type.
\Rule{
   $\epsilon$ = \syn{(Negate ($\epsilon_1$))}
\\  ((\environment $\ni$ \typeof{$\epsilon_1$}{\typ{$\tau$}}
\\ (($\tau$ = \typ{int}) $\vee$ ($\tau$ = \typ{float}))
}{
\environment $\ni$ \typeof{$\epsilon$}{{$\tau$}}
}{rule_NegateType}

\subsubsection{Typing String Expressions}
Rule \ref{rule_ConcatType} states that atomic string operations
such as \syn{Concat} can only be applied to arguments of type \typ{string} yielding a
result of type \typ{string}.

\Rule{
   $\epsilon$ = \syn{(Concat  ($\epsilon_1$ $\epsilon_2$))},
\\  \environment $\ni$ \typeof{$\epsilon_1$}{\typ{string}},
\\  \environment $\ni$ \typeof{$\epsilon_2$}{\typ{string}},
}{
\environment $\ni$ \typeof{$\epsilon$}{{\typ{string}}}
}{rule_ConcatType}


\subsection{Typing Aggregates}

Because aggregates operator over the whole bag rather than individual tuples the inputs to aggregate expressions are collected together before applying the aggregate operator.
\Rule{
$\epsilon$ = \syn{(Collection ($\epsilon_1$))},
\\   \environment $\ni$ \typeof{$\epsilon_1$}{$\tau$}
}{
\environment $\ni$ \typeof{$\epsilon$}{\typ{\bagof{$\tau$}}}
}{rule_CollectionType}

Rule \ref{rule_CountType} states that the type of an \syn{Count} on a bag of any type is always an \typ{int}.
\Rule{
$\epsilon$ = \syn{(Count ($\epsilon_1$))},
\\   \environment $\ni$ \typeof{$\epsilon_1$}{\bagof{\_}}
}{
\environment $\ni$ \typeof{$\epsilon$}{\typ{int}}
}{rule_CountType}

Rule \ref{rule_AvgType} states that the type of an \syn{Avg} on a bag of numerical value is always a \typ{float}.
\Rule{
$\epsilon$ = \syn{(Avg ($\epsilon_1$))},
\\   ((\environment $\ni$ \typeof{$\epsilon_1$}{\typ{\bagof{int}}}) $\vee$ (\environment $\ni$ \typeof{$\epsilon_1$}{\typ{\bagof{float}}}))
}{
\environment $\ni$ \typeof{$\epsilon$}{\typ{float}}
}{rule_AvgType}

Rule \ref{rule_MinIntType} states that the type of a other mathematical functions such \syn{Min}, \syn{Max} or \syn{Sum} applied on a bag of a single type is an instance of that type, as long as the type is numerical.
\Rule{
($\epsilon$ = \syn{(Min ($\epsilon_1$))} $\vee$
 $\epsilon$ = \syn{(Max ($\epsilon_1$))} $\vee$
 $\epsilon$ = \syn{(Sum ($\epsilon_1$))}),
\\   \environment $\ni$ \typeof{$\epsilon_1$}{\typ{\bagof{$\tau$}}}
\\ ($\tau$ = \typ{int} $\vee$ $\tau$ = \typ{float})
}{
\environment $\ni$ \typeof{$\epsilon$}{\typ{$\tau$}}
}
{rule_MinIntType}
\subsection{Typing Predicate Expressions}

The type of a valid \syn{Predicate} will always be \typ{Boolean}, or to be formal will be a function which takes a tuple as input and returns a \typ{Boolean}.

Rule \ref{rule_Equals} states that any two expressions of the same type can be tested for equality or inequality.

\Rule{
(($\epsilon$ = \syn{(Equals ($\epsilon_1$ $\epsilon_2$))})
$\wedge$ ($\epsilon$ = \syn{(NotEquals ($\epsilon_1$ $\epsilon_2$))}))
\\   \environment $\ni$ \typeof{$\epsilon_1$}{$\tau_1$},
\\   \environment $\ni$ \typeof{$\epsilon_2$}{$\tau_2$},
\\ $\tau_1$ = $\tau_2$
}{
\environment $\ni$ \typeof{$\epsilon$}{\typ{Boolean}}
}{rule_Equals}

Rule \ref{rule_compareType} states that mathematical comparison expressions \syn{GreaterThan},

\noindent \syn{GreaterThanOrEquals}, \syn{LessThan} or \syn{LessThanOrEquals} are valid if both expressions being compared are of type \typ{int} or both are of type \typ{float}

\Rule{
((($\epsilon$ = \syn{(GreaterThan ($\epsilon_1$ $\epsilon_2$))})
$\wedge$ ($\epsilon$ = \syn{(LessThan ($\epsilon_1$ $\epsilon_2$))})
\\$\wedge$ ($\epsilon$ = \syn{(GreaterThanOrEquals ($\epsilon_1$ $\epsilon_2$))})
\\$\wedge$ ($\epsilon$ = \syn{(LessThanOrEquals ($\epsilon_1$ $\epsilon_2$))})),
\\   (((\environment $\ni$ \typeof{$\epsilon_1$}{\typ{int}}) 
    $\vee$ (\environment $\ni$ \typeof{$\epsilon_2$}{\typ{int}})) $\wedge$
   ((\environment $\ni$ \typeof{$\epsilon_1$}{\typ{float}}) 
    $\vee$ (\environment $\ni$ \typeof{$\epsilon_2$}{\typ{float}})))
}{
\environment $\ni$ \typeof{$\epsilon$}{\typ{Boolean}}
}{rule_compareType}

Rule \ref{rule_castCompareType} states that \syn{Equals}, \syn{NotEquals} and mathematical comparison expressions \syn{GreaterThan}, 
\syn{GreaterThanOrEquals}, \syn{LessThan} or \syn{LessThanOrEquals} where one expression has a type \typ{int} and the other has type \typ{float} are valid.
This assumes an implied type casting of the \typ{int} value to a \typ{float} value.

\Rule{
(($\epsilon$ = \syn{(Equals ($\epsilon_1$ $\epsilon_2$))}) 
$\wedge$ ($\epsilon$ = \syn{(NotEquals ($\epsilon_1$ $\epsilon_2$))}))
\\$\wedge$ ($\epsilon$ = \syn{(GreaterThan ($\epsilon_1$ $\epsilon_2$))})
\\$\wedge$ ($\epsilon$ = \syn{(GreaterThanOrEquals ($\epsilon_1$ $\epsilon_2$))})
\\$\wedge$ ($\epsilon$ = \syn{(LessThan ($\epsilon_1$ $\epsilon_2$))})
$\wedge$ ($\epsilon$ = \syn{(LessThanOrEquals ($\epsilon_1$ $\epsilon_2$))})),
\\   (((\environment $\ni$ \typeof{$\epsilon_1$}{\typ{int}}) 
    $\vee$ (\environment $\ni$ \typeof{$\epsilon_2$}{\typ{float}})) $\wedge$
   ((\environment $\ni$ \typeof{$\epsilon_1$}{\typ{float}}) 
    $\vee$ (\environment $\ni$ \typeof{$\epsilon_2$}{\typ{int}})))
}{
\environment $\ni$ \typeof{$\epsilon$}{\typ{Boolean}}
}{rule_castCompareType}

Rule \ref{rule_AndType} states that an \syn{And} or \syn{Or} is a \typ{Boolean} if all input are \typ{Boolean}.
\Rule{
(($\epsilon$ = \syn{(And ([$\epsilon_1$, \mdots, $\epsilon_n$]))}) 
    $\wedge$ ($\epsilon$ = \syn{(Or ([$\epsilon_1$, \mdots, $\epsilon_n$]))})),
\\  \environment $\ni$ \typeof{$\epsilon_1$}{\typ{Boolean}}, \mdots, \environment $\ni$ \typeof{$\epsilon_n$}{\typ{Boolean}}
}{
\environment $\ni$ \typeof{$\epsilon$}{\typ{Boolean}}
}{rule_AndType}

Rule \ref{rule_NotType} states that any \syn{Not} is a \typ{Boolean} as long as the input is a \typ{Boolean}.
\Rule{
$\epsilon$ = \syn{(Not ($\epsilon_1$))},
\\   \environment $\ni$ \typeof{$\epsilon_1$}{\typ{Boolean}},
}{
\environment $\ni$ \typeof{$\epsilon$}{\typ{Boolean}}
}{rule_NotType}

\subsection{Typing the Group By Clause}
Rule \ref{rule_GroupByType} shows that typing a \syn{GroupByList} consists of typing each \syn{Attribute} [$\epsilon_1$, \mdots, $\epsilon_n$].
Rule \ref{rule_AttributeType} which types a \syn{Attribute} also verifies that the attribute is used in the FROM clause.

The rule also declares a \syn{DeclGroupBy} which hold the \syn{Attribute} list.
This way any \syn{Attribute} in the Select, Having and Where clauses can be checked by rule \ref{rule_AttributeInGroupType}.
There will never be more than one \syn{DeclGroupBy} per environment as there is only one Group By statement in any query. 
Sub queries are typed in a separate environment which can have a separate \syn{DeclGroupBy}.

The type of a GroupByList is a tuple of the types of each of the attributes.

\Rule{
$\epsilon$ = \syn{(GroupByList ([$\epsilon_1$, \mdots, $\epsilon_n$]))},
\\  \environment $\ni$ \typeof{$\epsilon_1$}{$\tau_1$}, \mdots, \environment $\ni$ \typeof{$\epsilon_n$}{$\tau_n$},
}{
\environment $\ni$ \syn{(DeclGroupBy ([$\epsilon_1$, \mdots, $\epsilon_n$]))},
\\ \environment $\ni$ \typeof{$\epsilon$}{\tupleof{$\tau_1$, \mdots, $\tau_n$}}
}{rule_GroupByType}



