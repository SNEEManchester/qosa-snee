\section{Grammar to convert Query Tokens into an AST}
\label{sec_Grammar}
This section shows the \SNEEql grammar. 
The grammar takes a series of tokens as inputs and returns an AST (Abstract Syntax Tree).
The tokens are keywords, symbols and wrappers around integer values, float values and quoted strings with all other text assumed to be variable names.

The Haskell implementation uses the Happy parser generator so the grammar is shown in Happy syntax.

Each rule is represented in the format:

\begin{code} 
ruleName   : "keyword" otherRule       { HaskelConstuctor $2}
\end{code} 

Starting with the name of the grammar rule.
The colon represents the start of a rule while a vertical line "|" starts an alternative version.
The next part lists the tokens expected by this grammar rule.
Anything shown in quotes is a keyword, int, float, string and var represent integers, floats, quoted strings and variables names.
Any other names refers to another grammar rule.

The Haskell AST constructor used by Happy for each rule is shown in curly brackets. The \$n represents pass the value found in the nth position to the Constructor. The notion [\$1] means build a list with the first value as its only member while \$1 : \$3 means add the first value to the list returned by the third value.

The grammar included semantic sugar indicated by the   {- Semantic Sugar -} comment.
Whenever possible the grammar converts the semantic sugar format to the full format.

So for example the line:
\begin{code}
  | extentName  { Extent (ExtentName $1) (LocalName $1) }
\end{code}
takes an extent declaration where only the extent name is provided and uses the same var  the \syn{LocalName}.
This results in an Identical AST for the declaration "FROM Employee Employee" and its shorten form "FROM Employee".
  
To keep the typing and translation rules simple the parser distinguishes between lists and expressions without aggregation, those with aggregation and those which mix both aggregated and none aggregated attributes.
This allows the parser to distinguish between simple queries and aggregated queries, while allowing (but not forcing) group by queries to have mixed list and expressions.

Identify the type of expression while still allowing expression trees such as "avg(temp + 3 * 2) + max(pressure) - min(pressure)" required repeating a lot similar grammar rules.

Priority between arithmetic expressions is provided atomically by the declarations:
\begin{code}
%nonassoc ">" "<"
%left "+" "-"
%left "*" "/"
%left "^"
%left "="
%left NEG
\end{code}

\subsection{Actual Grammar file}
\begin{code}
\section{Grammar to convert Query Tokens into an AST}
\label{sec_Grammar}
This section shows the \SNEEql grammar. 
The grammar takes a series of tokens as inputs and returns an AST (Abstract Syntax Tree).
The tokens are keywords, symbols and wrappers around integer values, float values and quoted strings with all other text assumed to be variable names.

The Haskell implementation uses the Happy parser generator so the grammar is shown in Happy syntax.

Each rule is represented in the format:

\begin{code} 
ruleName   : "keyword" otherRule       { HaskelConstuctor $2}
\end{code} 

Starting with the name of the grammar rule.
The colon represents the start of a rule while a vertical line "|" starts an alternative version.
The next part lists the tokens expected by this grammar rule.
Anything shown in quotes is a keyword, int, float, string and var represent integers, floats, quoted strings and variables names.
Any other names refers to another grammar rule.

The Haskell AST constructor used by Happy for each rule is shown in curly brackets. The \$n represents pass the value found in the nth position to the Constructor. The notion [\$1] means build a list with the first value as its only member while \$1 : \$3 means add the first value to the list returned by the third value.

The grammar included semantic sugar indicated by the   {- Semantic Sugar -} comment.
Whenever possible the grammar converts the semantic sugar format to the full format.

So for example the line:
\begin{code}
  | extentName  { Extent (ExtentName $1) (LocalName $1) }
\end{code}
takes an extent declaration where only the extent name is provided and uses the same var  the \syn{LocalName}.
This results in an Identical AST for the declaration "FROM Employee Employee" and its shorten form "FROM Employee".
  
To keep the typing and translation rules simple the parser distinguishes between lists and expressions without aggregation, those with aggregation and those which mix both aggregated and none aggregated attributes.
This allows the parser to distinguish between simple queries and aggregated queries, while allowing (but not forcing) group by queries to have mixed list and expressions.

Identify the type of expression while still allowing expression trees such as "avg(temp + 3 * 2) + max(pressure) - min(pressure)" required repeating a lot similar grammar rules.

Priority between arithmetic expressions is provided atomically by the declarations:
\begin{code}
%nonassoc ">" "<"
%left "+" "-"
%left "*" "/"
%left "^"
%left "="
%left NEG
\end{code}

\subsection{Actual Grammar file}
\begin{code}
\section{Grammar to convert Query Tokens into an AST}
\label{sec_Grammar}
This section shows the \SNEEql grammar. 
The grammar takes a series of tokens as inputs and returns an AST (Abstract Syntax Tree).
The tokens are keywords, symbols and wrappers around integer values, float values and quoted strings with all other text assumed to be variable names.

The Haskell implementation uses the Happy parser generator so the grammar is shown in Happy syntax.

Each rule is represented in the format:

\begin{code} 
ruleName   : "keyword" otherRule       { HaskelConstuctor $2}
\end{code} 

Starting with the name of the grammar rule.
The colon represents the start of a rule while a vertical line "|" starts an alternative version.
The next part lists the tokens expected by this grammar rule.
Anything shown in quotes is a keyword, int, float, string and var represent integers, floats, quoted strings and variables names.
Any other names refers to another grammar rule.

The Haskell AST constructor used by Happy for each rule is shown in curly brackets. The \$n represents pass the value found in the nth position to the Constructor. The notion [\$1] means build a list with the first value as its only member while \$1 : \$3 means add the first value to the list returned by the third value.

The grammar included semantic sugar indicated by the   {- Semantic Sugar -} comment.
Whenever possible the grammar converts the semantic sugar format to the full format.

So for example the line:
\begin{code}
  | extentName  { Extent (ExtentName $1) (LocalName $1) }
\end{code}
takes an extent declaration where only the extent name is provided and uses the same var  the \syn{LocalName}.
This results in an Identical AST for the declaration "FROM Employee Employee" and its shorten form "FROM Employee".
  
To keep the typing and translation rules simple the parser distinguishes between lists and expressions without aggregation, those with aggregation and those which mix both aggregated and none aggregated attributes.
This allows the parser to distinguish between simple queries and aggregated queries, while allowing (but not forcing) group by queries to have mixed list and expressions.

Identify the type of expression while still allowing expression trees such as "avg(temp + 3 * 2) + max(pressure) - min(pressure)" required repeating a lot similar grammar rules.

Priority between arithmetic expressions is provided atomically by the declarations:
\begin{code}
%nonassoc ">" "<"
%left "+" "-"
%left "*" "/"
%left "^"
%left "="
%left NEG
\end{code}

\subsection{Actual Grammar file}
\begin{code}
\section{Grammar to convert Query Tokens into an AST}
\label{sec_Grammar}
This section shows the \SNEEql grammar. 
The grammar takes a series of tokens as inputs and returns an AST (Abstract Syntax Tree).
The tokens are keywords, symbols and wrappers around integer values, float values and quoted strings with all other text assumed to be variable names.

The Haskell implementation uses the Happy parser generator so the grammar is shown in Happy syntax.

Each rule is represented in the format:

\begin{code} 
ruleName   : "keyword" otherRule       { HaskelConstuctor $2}
\end{code} 

Starting with the name of the grammar rule.
The colon represents the start of a rule while a vertical line "|" starts an alternative version.
The next part lists the tokens expected by this grammar rule.
Anything shown in quotes is a keyword, int, float, string and var represent integers, floats, quoted strings and variables names.
Any other names refers to another grammar rule.

The Haskell AST constructor used by Happy for each rule is shown in curly brackets. The \$n represents pass the value found in the nth position to the Constructor. The notion [\$1] means build a list with the first value as its only member while \$1 : \$3 means add the first value to the list returned by the third value.

The grammar included semantic sugar indicated by the   {- Semantic Sugar -} comment.
Whenever possible the grammar converts the semantic sugar format to the full format.

So for example the line:
\begin{code}
  | extentName  { Extent (ExtentName $1) (LocalName $1) }
\end{code}
takes an extent declaration where only the extent name is provided and uses the same var  the \syn{LocalName}.
This results in an Identical AST for the declaration "FROM Employee Employee" and its shorten form "FROM Employee".
  
To keep the typing and translation rules simple the parser distinguishes between lists and expressions without aggregation, those with aggregation and those which mix both aggregated and none aggregated attributes.
This allows the parser to distinguish between simple queries and aggregated queries, while allowing (but not forcing) group by queries to have mixed list and expressions.

Identify the type of expression while still allowing expression trees such as "avg(temp + 3 * 2) + max(pressure) - min(pressure)" required repeating a lot similar grammar rules.

Priority between arithmetic expressions is provided atomically by the declarations:
\begin{code}
%nonassoc ">" "<"
%left "+" "-"
%left "*" "/"
%left "^"
%left "="
%left NEG
\end{code}

\subsection{Actual Grammar file}
\begin{code}
\input{SNEEqlHappy.y}
\end{code}  

\end{code}  

\end{code}  

\end{code}  
