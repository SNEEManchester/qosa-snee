\section{Translation rules into executable operators}
\label{sec_translating}
\texttt{Gives formal rules for translating from the query language to actual operators.
Includes specific rule of when to use the  relational, windowed or stream versions of each operators.}

\texttt{Again the rules be presented as formal equations here with the corresponding Haskell code left for the Appendix.}

\section{Translating a \SNEEql Query}
\subsection{Translating a Query}
\label{sec_wholeQueryTranslation}

This sections introduces the rules for converting an AST into a series of algebraic operators expressed in Haskell notation.

The translation rules do not include any optimisation as these will not affect the semantics of the query language.
Therefore \sem{CrossProduct} and \sem{Select} operators are used instead of \sem{Join}. 
No attempt is made to push down \sem{Select} and \sem{Project} operators.
Queries without a WHERE clause will have an implied predicate of \syn{TRUE} and therefore a \sem{SELECT} operator with \syn{TRUE} as its parameter.
 
\subsubsection{Translation Rules Format}
\newcounter{equation2}
\setcounter{equation2}{\value{equation}}
\setcounter{equation}{-1}

The rules for Translation are expressed in the same format as the typing rules (see Section \ref{sec_TypingRulesFormat}).
For example one of the rules for typing a \syn{Query} is:

\Rule{
$\epsilon$ = \syn{(Query ($\epsilon_1$ $\epsilon_2$ $\epsilon_3$  \_ ))},
\\  \environment' $\ni$ \typeof{$\epsilon_2$}{\bagof{\tupleof{\_}}},
\\ \environment' $\ni$ \tranof{$\epsilon_2$}{$\alpha$},
\\  $\epsilon_1$ = \syn{[(Projection ($\epsilon'_1$ $\epsilon''_1$)), \mdots, (Projection ($\epsilon'_n$ $\epsilon''_n$))]}
}{
\environment $\ni$ \tranof{$\epsilon$}{\sem{RelProject [$\epsilon'_1$, \mdots, $\epsilon'_n$] [$\epsilon''_1$, \mdots, $\epsilon''_1$] (RelSelect $\epsilon_3$ $\alpha$)}}
}{rule_Ex2}

Translation of an AST element into an algebra element is represented as \tranof{$\epsilon$}{$\alpha$}, where $\alpha$ represents an algebraic operator. The font \sem{Operator} is used for Haskell constructors of operators. 

As the above example shows the translation rules make use of the conclusion of the typing rules, so conclusions made there and validity checks are not repeated unless specifically required.

\setcounter{equation}{\value{equation2}}
%\input{SNEEqlTypes.lhs}

\subsubsection{Translating a FullQuery}
Rule \ref{rule_TranslateFullQuery} says that the translation of a \syn{FullQuery} is a \sem{Deliver} operator over the output of translating the query $\epsilon_1$. 

\Rule{
$\epsilon$ = \syn{(FullQuery ($\epsilon_1$))},
\\ \environment $\ni$ \tranof{$\epsilon_1$}{$\alpha$}
}{
\environment $\ni$ \tranof{$\epsilon$}{(\sem{Deliver} $\alpha$)}
}{rule_TranslateFullQuery}

Translating a \syn{Query} starts by typing the \syn{ExtentList} $\epsilon_2$.
This helps determines which translation rule to use and therefore which \sem{Project} and \sem{Select} operators to use.
The \syn{ExtentList} is then translated with the result becoming the input for the \sem{Select} operators.
The \sem{Select} operator takes the \syn{BooleanExpression} $\epsilon_3$ as its parameter. 

This \sem{Select} operator is then itself the input for a \sem{Project} or \sem{Aggregation} operator which takes the \syn{Expression} $\epsilon_1$ as its additional parameter. 

\subsubsection{Translating a Relational Query}

The traditional relational \syn{Query} where the type of the \syn{ExtentList} $\epsilon_2$ is of bag of tuples will use the Relational operators \sem{RelSelect} and \sem{RelProject}.

\Rule{
$\epsilon$ = \syn{(Query ($\epsilon_1$ $\epsilon_2$ $\epsilon_3$  \_ ))},
\\  \environment' $\ni$ \typeof{$\epsilon_2$}{\bagof{\tupleof{\_}}},
\\ \environment' $\ni$ \tranof{$\epsilon_2$}{$\alpha$},
\\  $\epsilon_1$ = \syn{[(Projection ($\epsilon'_1$ $\epsilon''_1$)), \mdots, (Projection ($\epsilon'_n$ $\epsilon''_n$))]}
}{
\environment $\ni$ \tranof{$\epsilon$}{\sem{RelProject [$\epsilon'_1$, \mdots, $\epsilon'_n$] [$\epsilon''_1$, \mdots, $\epsilon''_1$] (RelSelect $\epsilon_3$ $\alpha$)}}
}{rule_TranslateRelation}

The relational \syn{AggrgeationQuery} where the type of the \syn{ExtentList} $\epsilon_2$ is of bag of tuples and where the SELECT clause includes aggregation, uses the Relational operators \sem{RelSelect} and \sem{RelAggregation}.

\Rule{  
$\epsilon$ = \syn{(AggregationQuery ($\epsilon_1$ $\epsilon_2$ $\epsilon_3$  $\epsilon_4$))},
\\  \environment' $\ni$ \typeof{$\epsilon_2$}{\bagof{\tupleof{\_}}},
\\ \environment' $\ni$ \tranof{$\epsilon_2$}{$\alpha$},
\\  $\epsilon_1$ = \syn{[(Aggregation ($\epsilon'_1$ $\epsilon''_1$)), \mdots, (Aggregation ($\epsilon'_n$ $\epsilon''_n$))]}
}{
\environment $\ni$ \tranof{$\epsilon$}{\sem{RelAggregation [$\epsilon'_1$, \mdots, $\epsilon'_n$] [$\epsilon''_1$, \mdots, $\epsilon''_1$] (RelSelect $\epsilon_3$ $\alpha$)}}
}{rule_TranslateRelationAggregation}

\subsubsection{Translating a Window-Stream Query}

A \syn{Query} where the type of the \syn{ExtentList} $\epsilon_2$ is windows ({\streamof{\tupleof{int, \bagof{\tupleof{\_}}}}}) will use the window operators \sem{WinSelect} and \sem{WinProject}.

\Rule{
$\epsilon$ = \syn{(Query ($\epsilon_1$ $\epsilon_2$ $\epsilon_3$  $\epsilon_4$))},
\\  \environment' $\ni$ \typeof{$\epsilon_2$}{\streamof{\tupleof{\typ{int}, \bagof{\tupleof{\_}}}}},
\\ \environment' $\ni$ \tranof{$\epsilon_2$}{$\alpha$},
\\  $\epsilon_1$ = \syn{[(Projection ($\epsilon'_1$ $\epsilon''_1$)), \mdots, (Projection ($\epsilon'_n$ $\epsilon''_n$))]}
}{
\environment $\ni$ \tranof{$\epsilon$}{\sem{WinProject [$\epsilon'_1$, \mdots, $\epsilon'_n$] [$\epsilon''_1$, \mdots, $\epsilon''_1$] (WinSelect $\epsilon_3$ $\alpha$)}}
}{rule_TranslateWindow}

A \syn{AggregationQuery} where the type of the \syn{ExtentList}$\epsilon_2$ is windows ({\streamof{\tupleof{int, \bagof{\tupleof{\_}}}}}),and where the SELECT clause includes aggregation, will use the window operators \sem{WinSelect} and \sem{WinAggregation}.

\Rule{
$\epsilon$ = \syn{(AggregationQuery ($\epsilon_1$ $\epsilon_2$ $\epsilon_3$  $\epsilon_4$))},
\\  \environment' $\ni$ \typeof{$\epsilon_2$}{\streamof{\tupleof{\typ{int}, \bagof{\tupleof{\_}}}}},
\\ \environment' $\ni$ \tranof{$\epsilon_2$}{$\alpha$},
\\  $\epsilon_1$ = \syn{[(Aggregation ($\epsilon'_1$ $\epsilon''_1$)), \mdots, (Aggregation ($\epsilon'_n$ $\epsilon''_n$))]}
}{
\environment $\ni$ \tranof{$\epsilon$}{\sem{WinAggregation [$\epsilon'_1$, \mdots, $\epsilon'_n$] [$\epsilon''_1$, \mdots, $\epsilon''_1$] (WinSelect $\epsilon_3$ $\alpha_3$)}}
}{rule_TranslateWindowAggregation}

\subsubsection{Translating a Group By Query}
Rule \ref{rule_TranslateGroupByRelation} shows that translation of a relational \syn{GroupByQuery} begins the same as a normal query with translation of the \syn{ExtentList} $\epsilon_2$ and the addition of a \sem{RelSelect} operator based on the WHERE \syn{BooleanExpression} $\epsilon_3$.
However then a \sem{RelGrSelect} operator is added with the \syn{Attribute}s from the \syn{GroupByList} $\epsilon_5$.
This is followed by a \sem{RelGrSelect} based on the \syn{BooleanExpression} $\epsilon_5$ from the HAVING clause.
Similar to the WHERE clause this would be \syn{TRUE} if there is no HAVING clause.
Finally instead of a project or aggregation operator there is a \sem{RelUngroup} operator which forms the requested output tuples.

\Rule{
$\epsilon$ = \syn{(GroupByQuery ($\epsilon_1$ $\epsilon_2$ $\epsilon_3$ $\epsilon_4$ $\epsilon_5$ \_ ))},
\\  \environment' $\ni$ \typeof{$\epsilon_2$}{\bagof{\tupleof{\_}}},
\\ \environment' $\ni$ \tranof{$\epsilon_2$}{$\alpha$},
\\  $\epsilon_1$ = \syn{[(Projection ($\epsilon'_1$ $\epsilon''_1$)), \mdots, (Projection ($\epsilon'_n$ $\epsilon''_n$))]}
\\ $\epsilon_4$ = \syn{(GroupByList ([$\epsilon^g_1$, \mdots, $\epsilon^g_n$]))}
}{
\environment $\ni$ \tranof{$\epsilon$}{\sem{RelUngroup [$\epsilon'_1$, \mdots, $\epsilon'_n$] [$\epsilon''_1$, \mdots, $\epsilon''_1$] (RelGrSelect $\epsilon_5$ 
\\(RelGroupBy [$\epsilon^g_1$, \mdots, $\epsilon^g_n$] (RelSelect $\epsilon_3$ $\alpha$)))}}
}{rule_TranslateGroupByRelation}

Rule \ref{rule_TranslateGroupByWindow} which translates a \syn{GroupByQuery} with windows is similar to Rule \ref{rule_TranslateGroupByRelation} accept that the \syn{ExtentList} types to windows and the window version of all the operators is used.

\Rule{
$\epsilon$ = \syn{(GroupByQuery ($\epsilon_1$ $\epsilon_2$ $\epsilon_3$ $\epsilon_4$ $\epsilon_5$ \_ ))},
\\  \environment' $\ni$ \typeof{$\epsilon_2$}{\streamof{\tupleof{\typ{int}, \bagof{\tupleof{\_}}}}},
\\ \environment' $\ni$ \tranof{$\epsilon_2$}{$\alpha$},
\\  $\epsilon_1$ = \syn{[(Projection ($\epsilon'_1$ $\epsilon''_1$)), \mdots, (Projection ($\epsilon'_n$ $\epsilon''_n$))]}
\\ $\epsilon_4$ = \syn{(GroupByList ([$\epsilon^g_1$, \mdots, $\epsilon^g_n$]))}
}{
\environment $\ni$ \tranof{$\epsilon$}{\sem{WinUngroup [$\epsilon'_1$, \mdots, $\epsilon'_n$] [$\epsilon''_1$, \mdots, $\epsilon''_1$] (WinGrSelect $\epsilon_5$ 
\\(WinGroupBy [$\epsilon^g_1$, \mdots, $\epsilon^g_n$] (WinSelect $\epsilon_3$ $\alpha$)))}}
}{rule_TranslateGroupByWindow}

\subsubsection{Translating Tuple-Stream Queries}

A \syn{Query} where the type of the \syn{ExtentList} $\epsilon_2$ is tuple stream ({\streamof{\tupleof{int, int, \tupleof{\_}}}}) will use the stream operators \sem{StrSelect} and \sem{StrProject}.

\Rule{
$\epsilon$ = \syn{(Query ($\epsilon_1$ $\epsilon_2$ $\epsilon_3$  $\epsilon_4$))},
\\  \environment' $\ni$ \typeof{$\epsilon_2$}{\streamof{\tupleof{\typ{int}, \typ{int}, \tupleof{\_\ }}}},
\\ \environment' $\ni$ \tranof{$\epsilon_2$}{$\alpha$},
\\  $\epsilon_1$ = \syn{[(Projection ($\epsilon'_1$ $\epsilon''_1$)), \mdots, (Projection ($\epsilon'_n$ $\epsilon''_n$))]}
}{
\environment $\ni$ \tranof{$\epsilon$}{\sem{StreamProject [$\epsilon'_1$, \mdots, $\epsilon'_n$] [$\epsilon''_1$, \mdots, $\epsilon''_1$] (StreamSelect $\epsilon_3$ $\alpha_3$)}}
}{rule_TranslatedStream}

\subsubsection{Typing DStream, IStream, RStream Queries}

The translation rules for \syn{DStream}, \syn{IStream} and \syn{RStream} translate the inner query $\epsilon_1$ and use the resulting operator $\alpha_1$ as input for a \sem{DStreamOp}, \sem{IStreamOp} or \sem{RStreamOp} respectively.

\Rule{$\epsilon$ = \syn{(DStream ($\epsilon_1$))},
\\ \environment $\ni$ \tranof{$\epsilon_1$}{$\alpha_1$}
}{
\environment $\ni$ \tranof{$\epsilon$}{\sem{DStreamOp} ($\alpha_1$)}
}{rule_TranslateDStream}

\Rule{$\epsilon$ = \syn{(IStream ($\epsilon_1$))},
\\ \environment $\ni$ \tranof{$\epsilon_1$}{$\alpha_1$}
}{
\environment $\ni$ \tranof{$\epsilon$}{\sem{IStreamOp} ($\alpha_1$)}
}{rule_TranslateIStream}

\Rule{$\epsilon$ = \syn{(RStream ($\epsilon_1$))},
\\ \environment $\ni$ \tranof{$\epsilon_1$}{$\alpha_1$}
}{
\environment $\ni$ \tranof{$\epsilon$}{\sem{RStreamOp} ($\alpha_1$)}
}{rule_TranslateRStream}

\subsection{Translating the Extents in a Query}
\label{sec_extentListTraanslate}

Translating an \syn{ExtentList} consists of first translating the individual \syn{Extent}s and then where applicable combining then into \sem{CrossProduct} operators.

\subsubsection{Translating an Extent}
The Translation of an \syn{Extent} depends on the \syn{DeclExt} declaration made of the \syn{ExtentName} $\epsilon_1$. 
 
Stored Relations declared with \syn{StoredToken} will be access with a \sem{OneOffScan} operator.

\Rule{
$\epsilon$ = \syn{(Extent ($\epsilon_1$ $\epsilon_2$))},
\\ \environment $\ni$ \syn{(DeclExt ($\epsilon_1$ StoredToken \_\ ))}, 
}{
\environment $\ni$ \tranof{$\epsilon$}{\sem{OneOffScan} ($\epsilon_1$ $\epsilon_2$)}
}{rule_TranslateStored}

Stream of tuples declared with \syn{PushedToken} will be received by a \sem{StrReceive} operator.

\Rule{
$\epsilon$ = \syn{(Extent ($\epsilon_1$ $\epsilon_2$))},
\\ \environment $\ni$ \syn{(DeclExt ($\epsilon_1$ PushedToken \_\ ))}, 
}{
\environment $\ni$ \tranof{$\epsilon$}{\sem{StrReceive} ($\epsilon_1$ $\epsilon_2$)}
}{rule_TranslatePushed}

Sensor data declared with \syn{SensedToken} will be acquired with a \sem{StrAcquire} operator.
The logical version of \sem{StrAcquire} does not included the acquisition rate an sites parameters that the distributed physical versions will have. 

\Rule{
$\epsilon$ = \syn{(Extent ($\epsilon_1$ $\epsilon_2$))},
\\ \environment $\ni$ \syn{(DeclExt ($\epsilon_1$ SensedToken \_\ ))}, 
}{
\environment $\ni$ \tranof{$\epsilon$}{\sem{StrAcquire} ($\epsilon_1$ $\epsilon_2$)}
}{rule_TranslateSensed}
              
The translation rule for \syn{WindowedExtent} with a \syn{TimeWindowDef} (rule \ref{rule_TranslateTimeWindow}) translates the inner extent $\epsilon_1$ and uses that as input operator  for a \sem{TimeWindow} with the extra parameters of the windowScope $\epsilon_2$ (from and to values) the slide $\epsilon_3$ in Ticks.         

\Rule{
$\epsilon$ = \syn{(WindowedExtent ($\epsilon_1$ (TimeWindowDef $\epsilon_2$ $\epsilon_3$)))},
\\ \environment $\ni$ \tranof{$\epsilon_1$}{$\alpha$}
}{
\environment $\ni$ \tranof{$\epsilon$}{\sem{TimeWindow} ($\epsilon_2$ $\epsilon_3$ $\alpha$)}
}{rule_TranslateTimeWindow}

The translation rule for \syn{WindowedExtent} with a \syn{RowWindowDef} (rule \ref{rule_TranslateRowWindow}) translates the inner extent $\epsilon_1$ and uses that as input operator  for a \sem{RowWindow} with the extra parameters of the windowScope $\epsilon_2$ (from and to values) the slide $\epsilon_3$ as an Int Index.         
\Rule{
$\epsilon$ = \syn{(WindowedExtent ($\epsilon_1$ (RowWindowDef $\epsilon_2$ $\epsilon_3$)))},
\\ \environment $\ni$ \tranof{$\epsilon_1$}{$\alpha$}
}{
\environment $\ni$ \tranof{$\epsilon$}{\sem{RowWindow} ($\epsilon_2$ $\epsilon_3$ $\alpha$)}
}{rule_TranslateRowWindow}

The translation rule for \syn{WindowedExtent} with a \syn{InputWindowDef} (rule \ref{rule_TranslateInputWindow}) translates the inner extent $\epsilon_1$ and uses that as input operator  for a \sem{InputWindow} with the extra parameters of the windowScope $\epsilon_2$ (from and to values).

Neither \syn{InputWindowDef} nor \sem{InputWindow} include a slide.
\Rule{
$\epsilon$ = \syn{(WindowedExtent ($\epsilon_1$ (InputWindowDef $\epsilon_2$)))},
\\ \environment $\ni$ \tranof{$\epsilon_1$}{$\alpha$}
}{
\environment $\ni$ \tranof{$\epsilon$}{\sem{InputWindow} ($\epsilon_2$ $\alpha$)}
}{rule_TranslateInputWindow}

The rules for Translation of a \syn{SubQuery} first type the inner query $\epsilon_1$.
The type returned determines which rule and therefore which \sem{Rename} operator will used.
The inner query $\epsilon_1$ is then translated $\alpha$ with this being used as the input to a \sem{Rename} operator with the \syn{LocalName} $\epsilon_3$ as an additional parameter.

\syn{SubQuery} where the type of the inner query $\epsilon_1$ is of bag of tuples will use the Relational rename operator \sem{RelRename}.

\Rule{
$\epsilon$ = \syn{(Subquery ($\epsilon_1$ \_ $\epsilon_3$))},
\\  \environment $\ni$ \typeof{$\epsilon_2$}{\bagof{\tupleof{\_}}},
\\ \environment $\ni$ \tranof{$\epsilon_1$}{$\alpha$}
}{
\environment $\ni$ \tranof{$\epsilon$}{\sem{RelRename} $\epsilon_3$ $\alpha$}
}{rule_TranslateRelSubQuery}

\syn{SubQuery} where the type of the inner query $\epsilon_1$ is windows ({\streamof{\tupleof{int, \bagof{\tupleof{\_}}}}}) will use the window rename operator \sem{WinRename}.

\Rule{
$\epsilon$ = \syn{(Subquery ($\epsilon_1$ \_ $\epsilon_3$))},
\\ \environment $\ni$ \typeof{$\epsilon_1$}{\streamof{\tupleof{\typ{int}, \bagof{\tupleof{\_}}}}},
\\ \environment $\ni$ \tranof{$\epsilon_1$}{$\alpha$}
}{
\environment $\ni$ \tranof{$\epsilon$}{\sem{WinRename} $\epsilon_3$ $\alpha$}
}{rule_TranslateWinSubQuery}

\syn{SubQuery} where the type of the inner query $\epsilon_1$ is tuple stream ({\streamof{\tupleof{int, int, \tupleof{\_}}}}) will use the stream rename operator \sem{StrRename}.

\Rule{
$\epsilon$ = \syn{(Subquery ($\epsilon_1$ \_ $\epsilon_3$))},
\\  \environment' $\ni$ \typeof{$\epsilon_2$}{\streamof{\tupleof{\typ{int}, \typ{int}, \tupleof{\_\ }}}},
\\ \environment $\ni$ \tranof{$\epsilon_1$}{$\alpha$}
}{
\environment $\ni$ \tranof{$\epsilon$}{\sem{StrRename} $\epsilon_3$ $\alpha$}
}{rule_TranslateStrSubQuery}

\subsubsection{Queries with a single Extent}

Rule \ref{rule_SingleExtent} says that when there is only one \syn{ExtentInstruction} in the FROM clause then the operator for that extent is the operator for the whole \syn{ExtentList}
\Rule{
$\epsilon$ = \syn{(ExtentList [$\epsilon_1$])},
\environment $\ni$ \tranof{$\epsilon$}{$\alpha_1$}
}{
\environment $\ni$ \tranof{$\epsilon$}{$\alpha_1$}
}{rule_SingleExtent}

\subsubsection{Product of Relations}

Rule \ref{rule_ProdRelTranslation} states that an \syn{ExtentList}, where all
extents are of collection type relation, will use nested relational cross product operators \sem{RelCrossProduct}.

\Rule{
$\epsilon$ = \syn{(ExtentList [$\epsilon_1$, \mdots, $\epsilon_n$])},
\\  \environment $\ni$ \typeof{$\epsilon_1$}{$\tau_1$}, \mdots, \environment $\ni$ \typeof{$\epsilon_n$}{$\tau_n$},
\\  $\tau_1$ = \typ{\bagof{\tupleof{\_}}},
\\  $\vdots$
\\  $\tau_n$ = \typ{\bagof{\tupleof{\_}}},
\\  \environment $\ni$ \tranof{$\epsilon_1$}{$\alpha_1$}, \mdots, \environment $\ni$ \tranof{$\epsilon_n$}{$\alpha_n$},
}{
\environment $\ni$ \tranof{$\epsilon$}{\sem{RelCrossProduct} ($\alpha_1$ \mdots (\sem{RelCrossProduct} ($\alpha_{(n-1)}$ $\alpha_n$)))}
}{rule_ProdRelTranslation}

\subsubsection{Product of Window Streams}

Rule \ref{rule_ProdWinType} states that an \syn{ExtentList}, where all
extents are of collection type window-stream,  will use nested window cross product operators \sem{WinCrossProduct}.

\Rule{
$\epsilon$ = \syn{(ExtentList [$\epsilon_1$, \mdots, $\epsilon_n$])},
\\  \environment $\ni$ \typeof{$\epsilon_1$}{$\tau_1$}, \mdots, \environment $\ni$ \typeof{$\epsilon_n$}{$\tau_n$}
\\  $\tau_1$ = \typ{\streamof{\tupleof{\typ{int}, \bagof{\tupleof{\_}}}}},
\\  $\vdots$
\\  $\tau_n$ = \typ{\streamof{\tupleof{\typ{int}, \bagof{\tupleof{\_}}}}},
\\  \environment $\ni$ \tranof{$\epsilon_1$}{$\alpha_1$}, \mdots, \environment $\ni$ \tranof{$\epsilon_n$}{$\alpha_1$}
}{
\environment $\ni$ \tranof{$\epsilon$}{\sem{WinCrossProduct} ($\alpha_1$ \mdots (\sem{WinCrossProduct} ($\alpha_{(n-1)}$ $\alpha_n$)))}
}{rule_ProdWinTranslation}

\subsubsection{Product of Multiple Relations and a Stream}

Rule \ref{rule_TranslateProdRelStream} applies to \syn{ExtentList},
where all extents but one have collection type relation with the last
being of collection type tuple-stream.
All operators except the last ($\alpha_1$ to $\alpha_{(n-1)}$) are combined using nested relational cross product operators \sem{RelCrossProduct}.
The result is combined with the stream operator $\alpha_n$ using the mixed cross product operator \sem{StrRelCrossProduct}.

\Rule{
$\epsilon$ = \syn{(ExtentList [$\epsilon_1$, \mdots, $\epsilon_n$])},
\\  \environment $\ni$ \typeof{$\epsilon_1$}{$\tau_1$}, \mdots, \environment $\ni$ \typeof{$\epsilon_n$}{$\tau_n$}
\\  $\tau_1$ = \typ{\streamof{\tupleof{\typ{int}, \typ{int}, \tupleof{\_}}}},
\\  $\tau_2$ = \typ{\bagof{\tupleof{\_}}},
\\  $\vdots$
\\  $\tau_{n}$ = \typ{\bagof{\tupleof{\_}}},
\\  \environment $\ni$ \tranof{$\epsilon_1$}{$\alpha_1$}, \mdots, \environment $\ni$ \tranof{$\epsilon_n$}{$\alpha_n$},
\\ $\alpha'_1$ = {\sem{RelCrossProduct} ($\alpha_2$ \mdots (\sem{RelCrossProduct} ($\alpha_{(n-1)}$ $\alpha_n$)))}
}{
\environment $\ni$ \tranof{$\epsilon$}{\sem{StrRelCrossProduct} ($\alpha_1$ $\alpha'_1$)}
}{rule_TranslateProdRelStream}

\subsubsection{Product of Multiple Relations and Window-Streams}

Rule \ref{rule_TranslateProdRelStream} applies to \syn{ExtentList},
where some extents are of collection type relation and the rest are of collection
type window-stream.
The relational operators ($\alpha_1$ to $\alpha_n$) are combined using nested relational cross product operators \sem{RelCrossProduct} ($\alpha'_1$)
The window operators ($\alpha_{(n+1)}$ to $\alpha_{(n+m)}$) are combined using nested window cross product operators \sem{WinCrossProduct} ($\alpha'_1$)
The result of the above two are combined the mixed cross product operator \sem{WinRelCrossProduct}.

\Rule{
$\epsilon$ = \syn{(ExtentList [$\epsilon_1$, \mdots, $\epsilon_n$])},
\\  \environment $\ni$ \typeof{$\epsilon_1$}{$\tau_1$}, \mdots, \environment $\ni$ \typeof{$\epsilon_n$}{$\tau_n$},
\\  $\tau_1$ = \typ{\streamof{\tupleof{\typ{int}, \bagof{\tupleof{\_}}}}},
\\  $\vdots$
\\  $\tau_{n}$ = \typ{\streamof{\tupleof{\typ{int}, \bagof{\tupleof{\_}}}}},
\\  $\tau_{n+1}$ = \typ{\bagof{\tupleof{\_}}},
\\  $\vdots$
\\  $\tau_{n+m}$ = \typ{\typ{int}, \bagof{\tupleof{\_}}},
\\  \environment $\ni$ \tranof{$\epsilon_1$}{$\alpha_1$}, \mdots, \environment $\ni$ \tranof{$\epsilon_{n+m}$}{$\alpha_{n+m}$},
\\ $\alpha'_1$ = {\sem{WinCrossProduct} ($\alpha_1$ \mdots (\sem{WinCrossProduct} ($\alpha_{n-1}$ $\alpha_{n}$)))},
\\ $\alpha'_2$ = {\sem{RelCrossProduct} ($alpha_{n+1}$ \mdots (\sem{RelCrossProduct} ($\alpha_{{n+m-1}}$ $\alpha_{n+m}$)))}
}{
\environment $\ni$ \tranof{$\epsilon$}{\sem{WinRelCrossProduct} ($\alpha'_1$ $\alpha'_2$)}
}{rule_TranslateProdRelWin}

\section{Semantics in Haskell Only}
\input{SNEEqlEvaluate.lhs}
